Recently, I've come back again to one of the prominent articles in area of polynomials, power sums etc.
That is \textit{Johann Faulhaber and sums of powers}~\cite{knuth1993johann}.
Indeed, a great source to get piece of mind in mathematics.
What was the thing occupied my attention is an odd power identity in terms of Binomial coefficients and
Central factorial numbers, in particular page 9
\begin{align*}
    n^1 &= \binom{n}{1} \\
    n^3 &= 6 \binom{n+1}{3} + \binom{n}{1} \\
    n^5 &= 120 \binom{n+2}{5} + 30 \binom{n+1}{3} + \binom{n}{1}
\end{align*}

What was particularly interesting is that the well known identity in terms of triangular numbers
and finite differences of cubes becomes clear and obvious, e.g
\begin{align*}
    \Delta n^3 = (n+1)^3 - n^3
    =  6 \binom{n+1}{2} + \binom{n}{0}
\end{align*}
where $\binom{n+1}{2}$ are triangular numbers.
Because $\left[ 6 \binom{n+2}{3} + \binom{n+1}{1} \right] - \left[ 6 \binom{n+1}{3} + \binom{n}{1} \right]$
simplifies by means of recurrence relation of binomial coefficients.
Moreover, the concept above allows to reach $N$-order power sum $\sum^N k^{2m+1}$
or finite difference $\Delta^N k^{2m+1}$ of odd powers simply by changing
binomial coefficients indexes.
Quite strong and impressive.

However, unexpected odd power identity appears when we notice that triangular
numbers in $\Delta n^3 = (n+1)^3 - n^3 =  6 \binom{n+1}{2} + \binom{n}{0}$
are equivalent to the sum of first $n$ non-negative integers
\begin{align*}
    \binom{n+1}{2} = \sum_{k=0}^{n} k
\end{align*}
Which leads to identity in terms of finite differences of cubes
\begin{align*}
    \Delta n^3 = (n+1)^3 - n^3 = 1 + 6 \sum_{k=0}^{n} k
\end{align*}
It is obvious that $n^3$ evaluates to the sum of its $n-1$ finite differences, so that
\begin{align}
    \label{eq:cubes-compact-form}
    n^3 = \sum_{k=0}^{n-1} \Delta k^3 = \sum_{k=0}^{n-1} \left( 1 + 6 \sum_{t=0}^{k} t \right)
\end{align}
In its explicit form
\begin{align}
    \label{eq:cubes-explicit-form}
    n^3 &= [1+6\cdot0]+[1+6\cdot0+6\cdot1]+[1+6\cdot0+6\cdot1+6\cdot2] \\
    &+ \cdots + [1+6\cdot0+6\cdot1+6\cdot2+\cdots+6\cdot(n-1)] \nonumber
\end{align}
We could use Faulhaber's formula on $\sum_{t=0}^{k} t$ in equation~\eqref{eq:cubes-compact-form},
which leads to well known and expected identity in cubes $n^3 = \sum_{k=0}^{n-1} \sum_{t=0}^{2} \binom{3}{t} k^t$.

Instead, let's rearrange the terms in~\eqref{eq:cubes-explicit-form} to get
\begin{align}
    \label{eq:rearrange}
    n^3 = n &+ [(n-0) \cdot 6 \cdot 0] + [(n-1)\cdot6\cdot1] + [(n-2)\cdot6\cdot2] \\
    &+ \cdots + [(n-k)\cdot 6 \cdot k] + \cdots + [1\cdot6\cdot(n-1)] \nonumber
\end{align}
By applying compact sigma sum notation yields an identity for cubes $n^3$
\begin{align*}
    n^3 = n + \sum_{k=0}^{n-1} 6k(n-k)
\end{align*}
The term $n$ in the sum above can be moved under sigma notation, because there is exactly $n$ iterations, therefore
\begin{align}
    \label{eq:cubes-sigma-notation}
    n^3 = \sum_{k=0}^{n-1} 6k(n-k) + 1
\end{align}
By inspecting the expression $6k(n-k) + 1$ we iterate under summation,
we can notice that it is symmetric over $k$, let be $T(n,k) = 6k(n-k) + 1$, then
\begin{align*}
    T(n,k) = T(n,n-k)
\end{align*}
This symmetry allows us to alter summation bounds again, so that
\begin{align}
    \label{eq:cubes-sigma-alter-summation-bounds}
    n^3 = \sum_{k=1}^{n} 6k(n-k) + 1
\end{align}
Assume that polynomial identities
$n^3 = \sum_{k=0}^{n-1} 6k(n-k) + 1$ and $n^3 = \sum_{k=1}^{n} 6k(n-k) + 1$
have explicit form as follows
\begin{align*}
    n^3 = \sum_{k} A_{1,1} k^1(n-k)^1 + A_{1,0} k^0(n-k)^0
\end{align*}
where $A_{1,1} = 6$ and $A_{1,0} = 1$, respectively.

It could be generalized even further, for every odd power $2m+1$, giving a set of real coefficients
$A_{m,0}, A_{m,1},A_{m,2},A_{m,3}, \ldots, A_{m,m}$ such that
\begin{align}
    \label{eq:odd-power-generalized}
    n^{2m+1} = \sum_{k=1}^{n} A_{m,0} k^0 (n-k)^0 + A_{m,1} k^1 (n-k)^1
    + \cdots + A_{m,m} k^m (n-k)^m
\end{align}
Leading to numerous polynomial identities, including its compact form
\begin{align*}
    n^{2m+1} = \sum_{r=0}^{m} \sum_{k=1}^{n} A_{m,r} k^r (n-k)^r; \quad n^{2m+1} = \sum_{r=0}^{m} \sum_{k=0}^{n-1} A_{m,r} k^r (n-k)^r
\end{align*}
For example,
\begin{align*}
    n^3 &= \sum_{k=1}^{n} 6k(n-k) + 1 \\
    n^5 &= \sum_{k=1}^{n} 30k^2(n-k)^2 + 1 \\
    n^7 &= \sum_{k=1}^{n} 140 k^3 (n-k)^3 - 14k(n-k) + 1 \\
    n^9 &= \sum_{k=1}^{n} 630 k^4(n-k)^4 - 120k(n-k) + 1 \\
    n^{11} &= \sum_{k=1}^{n} 2772 k^5(n-k)^5 + 660 k^2(n-k)^2 - 1386k(n-k) + 1
\end{align*}

Table of these coefficients

\begin{table}[H]
    \begin{center}
        \setlength\extrarowheight{-6pt}
        \begin{tabular}{c|cccccccc}
            $m/r$ & 0 & 1       & 2      & 3      & 4   & 5    & 6     & 7 \\ [3px]
            \hline
            0     & 1 &         &        &        &     &      &       &       \\
            1     & 1 & 6       &        &        &     &      &       &       \\
            2     & 1 & 0       & 30     &        &     &      &       &       \\
            3     & 1 & -14     & 0      & 140    &     &      &       &       \\
            4     & 1 & -120    & 0      & 0      & 630 &      &       &       \\
            5     & 1 & -1386   & 660    & 0      & 0   & 2772 &       &       \\
            6     & 1 & -21840  & 18018  & 0      & 0   & 0    & 12012 &       \\
            7     & 1 & -450054 & 491400 & -60060 & 0   & 0    & 0     & 51480
        \end{tabular}
    \end{center}
    \caption{Coefficients $\coeffA{m}{r}$. See OEIS sequences
    ~\cite{oeis_numerators_of_the_coefficient_a_m_r,oeis_denominators_of_the_coefficient_a_m_r}.}
    \label{tab:table_of_coefficients_a}
\end{table}

Recurrence relation for $A_{m,r}$ is given by~\cite{alekseyev2018mathoverflow}.

\section{Questions}\label{sec:questions}

\begin{question}
    The algorithm we used to obtain identities for
    cubes~\eqref{eq:cubes-sigma-notation} and~\eqref{eq:cubes-sigma-alter-summation-bounds}
    is quite simple, if not naive.
    I believe it should be discussed in mathematical literature, as well as identity
    that gives a set of real coefficients $A_{m,r}$ such that
    \begin{align*}
        n^{2m+1} = \sum_{k=1}^{n} A_{m,0} k^0 (n-k)^0 + A_{m,1} (n-k)^1
        + \cdots + A_{m,m} k^m (n-k)^m
    \end{align*}
    However, I was not able to find any references that in particular mention coefficients $A_{m,r}$,
    which is one of open questions.
\end{question}

\begin{question}
    Can we consider the process of obtaining the identities~\eqref{eq:cubes-sigma-notation} and~\eqref{eq:cubes-sigma-alter-summation-bounds}
    as an interpolation technique?
\end{question}

\begin{question}
    If the question (2.2) is true, can we consider equation~\eqref{eq:odd-power-generalized} as an interpolation technique too?
\end{question}
